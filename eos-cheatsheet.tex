% -*- coding: utf-8; -*-
% \RequirePackage[l2tabu,orthodox]{nag}
\documentclass[8pt]{beamer}
\mode<presentation>
{
  \usetheme{EOS}
}
\graphicspath{{images/}}

\usepackage{minted}
\newmint{bash}{}

% Just loading fontspec should cause LuaLaTeX to pick a font that can display most Unicode characters.
\usepackage{fontspec}

\usepackage{tabu}

\usepackage{microtype}
% Uncomment if using references.
% \usepackage{csquotes}
% \usepackage[backend=biber,style=authoryear]{biblatex}
% \addbibresource{citations.bib}

\usepackage[orientation=landscape,size=a1,scale=1.4]{beamerposter}

% Uncomment if using dot2tex.
% \usepackage{dot2texi}
% \usepackage{tikz} % For dot2texi
% \usetikzlibrary{shapes,arrows} % For dot2texi
% \usepackage{hyperref}
% \hypersetup{hidelinks}

\newcommand{\mailtohref}[1]{\urlstyle{same}\href{mailto:#1}{\textless\nolinkurl{#1}\textgreater}}
\newcommand{\command}[1]{\textbf{\texttt{#1}}}

\title{Mastering EOS}
\author[Woodring \& Fisk]{Ira Woodring \mailtohref{woodriir@gvsu.edu} \\
  Sean Fisk \mailtohref{fiskse@mail.gvsu.edu}}
\institute[GVSU]{Grand Valley State University}

\begin{document}
\begin{frame}[fragile]{}
  \begin{columns}
    \begin{column}{0.33\textwidth}
      % \vfill
      % \begin{block}{\large Fontsizes}
      %   \centering
      %   {\tiny tiny}\par
      %   {\scriptsize scriptsize}\par
      %   {\footnotesize footnotesize}\par
      %   {\normalsize normalsize}\par
      %   {\large large}\par
      %   {\Large Large}\par
      %   {\LARGE LARGE}\par
      %   {\veryHuge veryHuge}\par
      %   {\VeryHuge VeryHuge}\par
      %   {\VERYHuge VERYHuge}\par
      % \end{block}
      \begin{block}{Inter-EOS Password-less SSH}
        {\scriptsize \inputminted[tabsize=2]{bash}{scripts/ssh.bash}}
        For more security, use a passphrase and \texttt{ssh-agent}.
      \end{block}
      \begin{block}{Run \texttt{make} faster by using all processors}
        \mint{bash}!make --jobs="$(grep '^processor' /proc/cpuinfo | wc --lines)"!
        % Emacs AUCTeX thinks we're in math mode because of the $ in the above code.
        % The one in this comment sets it straight.
      \end{block}
      \begin{block}{Multiple terminals}
        As an alternative to opening multiple terminals running \texttt{ssh} or multiple PuTTY windows, terminal multiplexers such as \texttt{screen} and \texttt{tmux} may be used. \texttt{tmux} is recommended. Type \command{tmux} once SSH'd to start it. \\
        \textbf{Default \texttt{tmux} bindings:} \\
        {\newcommand{\key}[1]{\texttt{C-b #1}}
          \textbf{\key{c}} means \command{Control + b}, then \command{c}.
        \begin{tabu} to 0.9\linewidth { X[2] X X[2] X }
          \hline
          New window & \key{c} & Kill window & \key{\&} \\ \hline
          Last window & \key{l} & Jump to window 4 & \key{4} \\ \hline
          Next window & \key{n} & Previous window & \key{p} \\ \hline
          Enter copy mode & \key{[} & Paste buffer & \key{]} \\ \hline
        \end{tabu} \\[0.5em]
        \textbf{Warning:} Detached sessions will be killed periodically. Please do not leave important jobs running.
      }
    \end{block}
    \end{column}
    \begin{column}{0.33\textwidth}
      \begin{block}{Version control systems}
        Version control systems are used to track changes to a set of files. Their use is considered professional software development practice.
        \begin{itemize}
        \item Distributed Version Control Systems (recommended)
          \begin{itemize}
          \item Git (git) \url{<git-scm.com>}
          \item Mercurial (hg) \url{<mercurial.selenic.com>}
          \end{itemize}
        \item Centralized Version Control Systems
          \begin{itemize}
          \item Subversion (svn) \url{<subversion.apache.org>}
          \item Concurrent Versions System (cvs) \url{<cvs.nongnu.org>}
          \end{itemize}
        \end{itemize}
      \end{block}
      \begin{block}{Installing software to your EOS account}
        Installs version 1.17.0 of aria2, a download utility similar to wget but with more features. This install is typical of many programs using the GNU build system. The option of note is the \texttt{--prefix} option to \texttt{./configure}. You may install to any directory you like, but \url{\~/.local} is a common convention. \\
        {\scriptsize \inputminted[tabsize=2]{bash}{scripts/install-aria2.bash}}
      \end{block}
    \end{column}
    \begin{column}{0.33\textwidth}
      \begin{block}{Downloading files}
        Download a copy of the W4 form.
        \begin{minted}[gobble=10]{bash}
          wget 'http://www.irs.gov/pub/irs-pdf/fw4.pdf'
          curl 'http://www.irs.gov/pub/irs-pdf/fw4.pdf' > index.html
        \end{minted}
      \end{block}
      \begin{block}{Miscellaneous}
        \begin{itemize}
        \item If a user is already logged on to a machine, press \command{Ctrl-Alt-Backspace} to kill the X server and log them out.
        \item Change your password with \command{passwd}.
        \item Check your quota with \command{quota --human-readable} or \command{quota -s}.
        \end{itemize}
      \end{block}
      \begin{block}{Using the clipboard}
        \begin{minted}[gobble=10]{bash}
          alias copy='xclip -selection clipboard -in'
          alias paste='xclip -selection clipboard -out'
          echo 'EOS Rocks' | copy
          copy eos.txt
          paste > eos.txt
        \end{minted}
      \end{block}
      \begin{block}{Software build systems}
        Using a build system offers an alternative to repetitively typing compiler commands and is considered professional practice.
        \begin{itemize}
        \item C/C++
          \begin{itemize}
          \item make \url{<gnu.org/s/make>}
          \item CMake \url{<cmake.org>}
          \item SCons \url{<scons.org>}
          \end{itemize}
        \item Java
          \begin{itemize}
          \item Ant \url{<ant.apache.org>}
          \item Maven \url{<maven.apache.org>}
          \item Buildr \url{<buildr.apache.org>}
          \end{itemize}
        \end{itemize}
      \end{block}
    \end{column}
  \end{columns}
\end{frame}

% Uncomment if using references.
% \printbibliography

\end{document}
