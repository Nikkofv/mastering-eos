\begin{block}{Manipulating the \texttt{PATH}}
  When you execute a command, the shell searches for the executable using the \texttt{PATH} environment variable. \texttt{PATH} should be set to a colon-delimited list of directories containing executables. To view your \texttt{PATH}, type:
  \begin{indented}
    \begin{bashcode}
      echo $PATH
    \end{bashcode}
  \end{indented}
  For instance, to add a local scripts directory (conventionally called \url{\~/bin}) to your \texttt{PATH}, run:
  \begin{indented}
    \begin{bashcode}
      export PATH=~/bin:$PATH
    \end{bashcode}
  \end{indented}
  Note that this causes the shell to find your scripts before system executables of the same name; however, this effect is usually desired. To make this setting persist, consider adding it to your \nolinkurl{~/.bash_profile}.
\end{block}
